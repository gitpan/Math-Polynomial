\documentclass{article}		% -*- Mode: LaTeX -*-

\usepackage{amsmath,verbatim,latexsym}

\setlength\parindent{0pt}
\setlength\parskip\medskipamount

\title{Polynomial Interpolation in Perl}
\author{Mats Kindahl\thanks{IAR Systems, Uppsala}}

\begin{document}

\maketitle

The {\bf Lagrange's formula} for $n$-point interpolation of a polynom
is
\begin{displaymath}
    A(x) = \sum_{k=0}^{n-1} y_k
        \frac{\prod_{j \neq k} (x - x_j)}
         {\prod_{j \neq k} (x_k - x_j)}
\end{displaymath}

This formula is taken from the excellent book \emph{Introduction to
Algorithms} by Thomas H.~Cormen, Charles E.~Leiserson, Ronald
L.~Rivest.

To create an algorithm that performs polynom interpolation in
$\Theta(n^2)$-time the trick is to compute the polynomial $B(x) =
\prod_{j=0}^{n-1} (x - x_j)$ in advance (which can be done in
$\Theta(n^2)$ time), and then for each $k$ compute $T(x) = B(x) / (x -
x_k)$ ($\Theta(n)$ time), divide the result by $\prod_{j \neq k} (x_k
- x_j)$ ($\Theta(n)$ time to compute the value, which can be done by
computing $T(x_k)$) and then accumulate the result into the final
polynomia $A(x)$ ($\Theta(n)$ time).
%
A more formal description of the algorithm is:

\bigskip
\hspace{2em}\begin{minipage}{0.7\hsize}
\begin{tabbing}
8. \=\kill
\textbf{Input:} 
A list of pairs $\left<(x_0,y_0),\ldots,(x_n,y_n)\right>$. 
\\*
%
\textbf{Output:} 
A polynomial $A(x)$ such that $A(x_i) = y_i$ for $0 \leq i \leq n$.
\\*[\medskipamount]
%
1.      \>Let $A(x) = 0$ (a zero degree polynomial) \\*
2.      \>Compute $B(x) = \prod_{j=0}^{n-1} (x - x_j)$ \\*
3.      \>For each $k = 0,\ldots,n-1$, repeat steps 3--5 \\*
4.      \>\kern2em Let $T(x) = B(x) / (x - x_k)$ \\*
5.      \>\kern2em Let $A(x) := A(x) + T(x) / T(x_k)$
\end{tabbing}
\end{minipage}
\bigskip

The time complexity of the algorithm above is: 
\begin{displaymath}
\begin{split}
   \Theta(n^2) + n \cdot \left[\Theta(n) + \Theta(n) + \Theta(n)\right] 
	&=\; \Theta(n^2) + 3n \Theta(n) \\
	&=\; \Theta(n^2) + \Theta(n^2) \\
	&=\; \mathbf{\Theta(n^2)}.
\end{split}
\end{displaymath}

\section*{Problems}

In this section we will consider some problem with polynomial
interpolation in general. (Maybe I'll add something about errors in
computations later.)

\subsection*{The polynomial gives bad values}

When doing polynomial interpolation on a set of points, the resulting
polynomial will swing widely if the real function is not a
polynomial. This is most prominent when trying to extrapolate a
non-polynomial with a polynomial, but it can happen even when trying
to evaluate the polynomial in the interpolated range.

For example, consider the function $f(x) = 1/(3x - 2x^2)$. By evaluating
it at some points, we get the following sequence of samples.

\begin{displaymath}
\begin{array}{r|r|r|r|r|r}
\strut x_i	&0.050	&0.100	&0.750	&1.400	&1.450 \\
\hline
\strut y_i	&6.897 	&3.571 	&0.889	&3.571	&6.897 \\
\end{array}
\end{displaymath}

\begin{figure}[t]
\begin{center}
% GNUPLOT: LaTeX picture
\setlength{\unitlength}{0.240900pt}
\ifx\plotpoint\undefined\newsavebox{\plotpoint}\fi
\sbox{\plotpoint}{\rule[-0.200pt]{0.400pt}{0.400pt}}%
\begin{picture}(1500,900)(0,0)
\font\gnuplot=cmr10 at 10pt
\gnuplot
\sbox{\plotpoint}{\rule[-0.200pt]{0.400pt}{0.400pt}}%
\put(176.0,203.0){\rule[-0.200pt]{303.534pt}{0.400pt}}
\put(176.0,68.0){\rule[-0.200pt]{0.400pt}{194.888pt}}
\put(176.0,68.0){\rule[-0.200pt]{4.818pt}{0.400pt}}
\put(154,68){\makebox(0,0)[r]{-5}}
\put(1416.0,68.0){\rule[-0.200pt]{4.818pt}{0.400pt}}
\put(176.0,203.0){\rule[-0.200pt]{4.818pt}{0.400pt}}
\put(154,203){\makebox(0,0)[r]{0}}
\put(1416.0,203.0){\rule[-0.200pt]{4.818pt}{0.400pt}}
\put(176.0,338.0){\rule[-0.200pt]{4.818pt}{0.400pt}}
\put(154,338){\makebox(0,0)[r]{5}}
\put(1416.0,338.0){\rule[-0.200pt]{4.818pt}{0.400pt}}
\put(176.0,473.0){\rule[-0.200pt]{4.818pt}{0.400pt}}
\put(154,473){\makebox(0,0)[r]{10}}
\put(1416.0,473.0){\rule[-0.200pt]{4.818pt}{0.400pt}}
\put(176.0,607.0){\rule[-0.200pt]{4.818pt}{0.400pt}}
\put(154,607){\makebox(0,0)[r]{15}}
\put(1416.0,607.0){\rule[-0.200pt]{4.818pt}{0.400pt}}
\put(176.0,742.0){\rule[-0.200pt]{4.818pt}{0.400pt}}
\put(154,742){\makebox(0,0)[r]{20}}
\put(1416.0,742.0){\rule[-0.200pt]{4.818pt}{0.400pt}}
\put(176.0,877.0){\rule[-0.200pt]{4.818pt}{0.400pt}}
\put(154,877){\makebox(0,0)[r]{25}}
\put(1416.0,877.0){\rule[-0.200pt]{4.818pt}{0.400pt}}
\put(176.0,68.0){\rule[-0.200pt]{0.400pt}{4.818pt}}
\put(176,23){\makebox(0,0){0}}
\put(176.0,857.0){\rule[-0.200pt]{0.400pt}{4.818pt}}
\put(344.0,68.0){\rule[-0.200pt]{0.400pt}{4.818pt}}
\put(344,23){\makebox(0,0){0.2}}
\put(344.0,857.0){\rule[-0.200pt]{0.400pt}{4.818pt}}
\put(512.0,68.0){\rule[-0.200pt]{0.400pt}{4.818pt}}
\put(512,23){\makebox(0,0){0.4}}
\put(512.0,857.0){\rule[-0.200pt]{0.400pt}{4.818pt}}
\put(680.0,68.0){\rule[-0.200pt]{0.400pt}{4.818pt}}
\put(680,23){\makebox(0,0){0.6}}
\put(680.0,857.0){\rule[-0.200pt]{0.400pt}{4.818pt}}
\put(848.0,68.0){\rule[-0.200pt]{0.400pt}{4.818pt}}
\put(848,23){\makebox(0,0){0.8}}
\put(848.0,857.0){\rule[-0.200pt]{0.400pt}{4.818pt}}
\put(1016.0,68.0){\rule[-0.200pt]{0.400pt}{4.818pt}}
\put(1016,23){\makebox(0,0){1}}
\put(1016.0,857.0){\rule[-0.200pt]{0.400pt}{4.818pt}}
\put(1184.0,68.0){\rule[-0.200pt]{0.400pt}{4.818pt}}
\put(1184,23){\makebox(0,0){1.2}}
\put(1184.0,857.0){\rule[-0.200pt]{0.400pt}{4.818pt}}
\put(1352.0,68.0){\rule[-0.200pt]{0.400pt}{4.818pt}}
\put(1352,23){\makebox(0,0){1.4}}
\put(1352.0,857.0){\rule[-0.200pt]{0.400pt}{4.818pt}}
\put(176.0,68.0){\rule[-0.200pt]{303.534pt}{0.400pt}}
\put(1436.0,68.0){\rule[-0.200pt]{0.400pt}{194.888pt}}
\put(176.0,877.0){\rule[-0.200pt]{303.534pt}{0.400pt}}
\put(176.0,68.0){\rule[-0.200pt]{0.400pt}{194.888pt}}
\put(1306,812){\makebox(0,0)[r]{$f(x) = 1/(3x - 2x^2)$}}
\put(1328.0,812.0){\rule[-0.200pt]{15.899pt}{0.400pt}}
\put(189,802){\usebox{\plotpoint}}
\multiput(189.58,760.63)(0.492,-12.731){21}{\rule{0.119pt}{9.967pt}}
\multiput(188.17,781.31)(12.000,-275.314){2}{\rule{0.400pt}{4.983pt}}
\multiput(201.58,492.94)(0.493,-3.906){23}{\rule{0.119pt}{3.146pt}}
\multiput(200.17,499.47)(13.000,-92.470){2}{\rule{0.400pt}{1.573pt}}
\multiput(214.58,400.20)(0.493,-1.964){23}{\rule{0.119pt}{1.638pt}}
\multiput(213.17,403.60)(13.000,-46.599){2}{\rule{0.400pt}{0.819pt}}
\multiput(227.58,352.88)(0.493,-1.131){23}{\rule{0.119pt}{0.992pt}}
\multiput(226.17,354.94)(13.000,-26.940){2}{\rule{0.400pt}{0.496pt}}
\multiput(240.58,324.82)(0.492,-0.841){21}{\rule{0.119pt}{0.767pt}}
\multiput(239.17,326.41)(12.000,-18.409){2}{\rule{0.400pt}{0.383pt}}
\multiput(252.58,305.80)(0.493,-0.536){23}{\rule{0.119pt}{0.531pt}}
\multiput(251.17,306.90)(13.000,-12.898){2}{\rule{0.400pt}{0.265pt}}
\multiput(265.00,292.92)(0.652,-0.491){17}{\rule{0.620pt}{0.118pt}}
\multiput(265.00,293.17)(11.713,-10.000){2}{\rule{0.310pt}{0.400pt}}
\multiput(278.00,282.93)(0.728,-0.489){15}{\rule{0.678pt}{0.118pt}}
\multiput(278.00,283.17)(11.593,-9.000){2}{\rule{0.339pt}{0.400pt}}
\multiput(291.00,273.93)(1.033,-0.482){9}{\rule{0.900pt}{0.116pt}}
\multiput(291.00,274.17)(10.132,-6.000){2}{\rule{0.450pt}{0.400pt}}
\multiput(303.00,267.93)(1.378,-0.477){7}{\rule{1.140pt}{0.115pt}}
\multiput(303.00,268.17)(10.634,-5.000){2}{\rule{0.570pt}{0.400pt}}
\multiput(316.00,262.93)(1.378,-0.477){7}{\rule{1.140pt}{0.115pt}}
\multiput(316.00,263.17)(10.634,-5.000){2}{\rule{0.570pt}{0.400pt}}
\multiput(329.00,257.94)(1.651,-0.468){5}{\rule{1.300pt}{0.113pt}}
\multiput(329.00,258.17)(9.302,-4.000){2}{\rule{0.650pt}{0.400pt}}
\multiput(341.00,253.95)(2.695,-0.447){3}{\rule{1.833pt}{0.108pt}}
\multiput(341.00,254.17)(9.195,-3.000){2}{\rule{0.917pt}{0.400pt}}
\multiput(354.00,250.95)(2.695,-0.447){3}{\rule{1.833pt}{0.108pt}}
\multiput(354.00,251.17)(9.195,-3.000){2}{\rule{0.917pt}{0.400pt}}
\put(367,247.17){\rule{2.700pt}{0.400pt}}
\multiput(367.00,248.17)(7.396,-2.000){2}{\rule{1.350pt}{0.400pt}}
\put(380,245.17){\rule{2.500pt}{0.400pt}}
\multiput(380.00,246.17)(6.811,-2.000){2}{\rule{1.250pt}{0.400pt}}
\put(392,243.17){\rule{2.700pt}{0.400pt}}
\multiput(392.00,244.17)(7.396,-2.000){2}{\rule{1.350pt}{0.400pt}}
\put(405,241.17){\rule{2.700pt}{0.400pt}}
\multiput(405.00,242.17)(7.396,-2.000){2}{\rule{1.350pt}{0.400pt}}
\put(418,239.67){\rule{3.132pt}{0.400pt}}
\multiput(418.00,240.17)(6.500,-1.000){2}{\rule{1.566pt}{0.400pt}}
\put(431,238.67){\rule{2.891pt}{0.400pt}}
\multiput(431.00,239.17)(6.000,-1.000){2}{\rule{1.445pt}{0.400pt}}
\put(443,237.67){\rule{3.132pt}{0.400pt}}
\multiput(443.00,238.17)(6.500,-1.000){2}{\rule{1.566pt}{0.400pt}}
\put(456,236.17){\rule{2.700pt}{0.400pt}}
\multiput(456.00,237.17)(7.396,-2.000){2}{\rule{1.350pt}{0.400pt}}
\put(469,234.67){\rule{2.891pt}{0.400pt}}
\multiput(469.00,235.17)(6.000,-1.000){2}{\rule{1.445pt}{0.400pt}}
\put(494,233.67){\rule{3.132pt}{0.400pt}}
\multiput(494.00,234.17)(6.500,-1.000){2}{\rule{1.566pt}{0.400pt}}
\put(507,232.67){\rule{3.132pt}{0.400pt}}
\multiput(507.00,233.17)(6.500,-1.000){2}{\rule{1.566pt}{0.400pt}}
\put(520,231.67){\rule{2.891pt}{0.400pt}}
\multiput(520.00,232.17)(6.000,-1.000){2}{\rule{1.445pt}{0.400pt}}
\put(481.0,235.0){\rule[-0.200pt]{3.132pt}{0.400pt}}
\put(545,230.67){\rule{3.132pt}{0.400pt}}
\multiput(545.00,231.17)(6.500,-1.000){2}{\rule{1.566pt}{0.400pt}}
\put(532.0,232.0){\rule[-0.200pt]{3.132pt}{0.400pt}}
\put(571,229.67){\rule{2.891pt}{0.400pt}}
\multiput(571.00,230.17)(6.000,-1.000){2}{\rule{1.445pt}{0.400pt}}
\put(558.0,231.0){\rule[-0.200pt]{3.132pt}{0.400pt}}
\put(596,228.67){\rule{3.132pt}{0.400pt}}
\multiput(596.00,229.17)(6.500,-1.000){2}{\rule{1.566pt}{0.400pt}}
\put(583.0,230.0){\rule[-0.200pt]{3.132pt}{0.400pt}}
\put(634,227.67){\rule{3.132pt}{0.400pt}}
\multiput(634.00,228.17)(6.500,-1.000){2}{\rule{1.566pt}{0.400pt}}
\put(609.0,229.0){\rule[-0.200pt]{6.022pt}{0.400pt}}
\put(698,226.67){\rule{3.132pt}{0.400pt}}
\multiput(698.00,227.17)(6.500,-1.000){2}{\rule{1.566pt}{0.400pt}}
\put(647.0,228.0){\rule[-0.200pt]{12.286pt}{0.400pt}}
\put(901,226.67){\rule{3.132pt}{0.400pt}}
\multiput(901.00,226.17)(6.500,1.000){2}{\rule{1.566pt}{0.400pt}}
\put(711.0,227.0){\rule[-0.200pt]{45.771pt}{0.400pt}}
\put(965,227.67){\rule{3.132pt}{0.400pt}}
\multiput(965.00,227.17)(6.500,1.000){2}{\rule{1.566pt}{0.400pt}}
\put(914.0,228.0){\rule[-0.200pt]{12.286pt}{0.400pt}}
\put(1003,228.67){\rule{3.132pt}{0.400pt}}
\multiput(1003.00,228.17)(6.500,1.000){2}{\rule{1.566pt}{0.400pt}}
\put(978.0,229.0){\rule[-0.200pt]{6.022pt}{0.400pt}}
\put(1029,229.67){\rule{2.891pt}{0.400pt}}
\multiput(1029.00,229.17)(6.000,1.000){2}{\rule{1.445pt}{0.400pt}}
\put(1016.0,230.0){\rule[-0.200pt]{3.132pt}{0.400pt}}
\put(1054,230.67){\rule{3.132pt}{0.400pt}}
\multiput(1054.00,230.17)(6.500,1.000){2}{\rule{1.566pt}{0.400pt}}
\put(1041.0,231.0){\rule[-0.200pt]{3.132pt}{0.400pt}}
\put(1080,231.67){\rule{2.891pt}{0.400pt}}
\multiput(1080.00,231.17)(6.000,1.000){2}{\rule{1.445pt}{0.400pt}}
\put(1092,232.67){\rule{3.132pt}{0.400pt}}
\multiput(1092.00,232.17)(6.500,1.000){2}{\rule{1.566pt}{0.400pt}}
\put(1105,233.67){\rule{3.132pt}{0.400pt}}
\multiput(1105.00,233.17)(6.500,1.000){2}{\rule{1.566pt}{0.400pt}}
\put(1067.0,232.0){\rule[-0.200pt]{3.132pt}{0.400pt}}
\put(1131,234.67){\rule{2.891pt}{0.400pt}}
\multiput(1131.00,234.17)(6.000,1.000){2}{\rule{1.445pt}{0.400pt}}
\put(1143,236.17){\rule{2.700pt}{0.400pt}}
\multiput(1143.00,235.17)(7.396,2.000){2}{\rule{1.350pt}{0.400pt}}
\put(1156,237.67){\rule{3.132pt}{0.400pt}}
\multiput(1156.00,237.17)(6.500,1.000){2}{\rule{1.566pt}{0.400pt}}
\put(1169,238.67){\rule{2.891pt}{0.400pt}}
\multiput(1169.00,238.17)(6.000,1.000){2}{\rule{1.445pt}{0.400pt}}
\put(1181,239.67){\rule{3.132pt}{0.400pt}}
\multiput(1181.00,239.17)(6.500,1.000){2}{\rule{1.566pt}{0.400pt}}
\put(1194,241.17){\rule{2.700pt}{0.400pt}}
\multiput(1194.00,240.17)(7.396,2.000){2}{\rule{1.350pt}{0.400pt}}
\put(1207,243.17){\rule{2.700pt}{0.400pt}}
\multiput(1207.00,242.17)(7.396,2.000){2}{\rule{1.350pt}{0.400pt}}
\put(1220,245.17){\rule{2.500pt}{0.400pt}}
\multiput(1220.00,244.17)(6.811,2.000){2}{\rule{1.250pt}{0.400pt}}
\put(1232,247.17){\rule{2.700pt}{0.400pt}}
\multiput(1232.00,246.17)(7.396,2.000){2}{\rule{1.350pt}{0.400pt}}
\multiput(1245.00,249.61)(2.695,0.447){3}{\rule{1.833pt}{0.108pt}}
\multiput(1245.00,248.17)(9.195,3.000){2}{\rule{0.917pt}{0.400pt}}
\multiput(1258.00,252.61)(2.695,0.447){3}{\rule{1.833pt}{0.108pt}}
\multiput(1258.00,251.17)(9.195,3.000){2}{\rule{0.917pt}{0.400pt}}
\multiput(1271.00,255.60)(1.651,0.468){5}{\rule{1.300pt}{0.113pt}}
\multiput(1271.00,254.17)(9.302,4.000){2}{\rule{0.650pt}{0.400pt}}
\multiput(1283.00,259.59)(1.378,0.477){7}{\rule{1.140pt}{0.115pt}}
\multiput(1283.00,258.17)(10.634,5.000){2}{\rule{0.570pt}{0.400pt}}
\multiput(1296.00,264.59)(1.378,0.477){7}{\rule{1.140pt}{0.115pt}}
\multiput(1296.00,263.17)(10.634,5.000){2}{\rule{0.570pt}{0.400pt}}
\multiput(1309.00,269.59)(1.033,0.482){9}{\rule{0.900pt}{0.116pt}}
\multiput(1309.00,268.17)(10.132,6.000){2}{\rule{0.450pt}{0.400pt}}
\multiput(1321.00,275.59)(0.728,0.489){15}{\rule{0.678pt}{0.118pt}}
\multiput(1321.00,274.17)(11.593,9.000){2}{\rule{0.339pt}{0.400pt}}
\multiput(1334.00,284.58)(0.652,0.491){17}{\rule{0.620pt}{0.118pt}}
\multiput(1334.00,283.17)(11.713,10.000){2}{\rule{0.310pt}{0.400pt}}
\multiput(1347.58,294.00)(0.493,0.536){23}{\rule{0.119pt}{0.531pt}}
\multiput(1346.17,294.00)(13.000,12.898){2}{\rule{0.400pt}{0.265pt}}
\multiput(1360.58,308.00)(0.492,0.841){21}{\rule{0.119pt}{0.767pt}}
\multiput(1359.17,308.00)(12.000,18.409){2}{\rule{0.400pt}{0.383pt}}
\multiput(1372.58,328.00)(0.493,1.131){23}{\rule{0.119pt}{0.992pt}}
\multiput(1371.17,328.00)(13.000,26.940){2}{\rule{0.400pt}{0.496pt}}
\multiput(1385.58,357.00)(0.493,1.964){23}{\rule{0.119pt}{1.638pt}}
\multiput(1384.17,357.00)(13.000,46.599){2}{\rule{0.400pt}{0.819pt}}
\multiput(1398.58,407.00)(0.493,3.906){23}{\rule{0.119pt}{3.146pt}}
\multiput(1397.17,407.00)(13.000,92.470){2}{\rule{0.400pt}{1.573pt}}
\multiput(1411.58,506.00)(0.492,12.731){21}{\rule{0.119pt}{9.967pt}}
\multiput(1410.17,506.00)(12.000,275.314){2}{\rule{0.400pt}{4.983pt}}
\put(1118.0,235.0){\rule[-0.200pt]{3.132pt}{0.400pt}}
\put(1306,767){\makebox(0,0)[r]{Sample}}
\put(1350,767){\raisebox{-.8pt}{\makebox(0,0){$\Diamond$}}}
\put(218,389){\raisebox{-.8pt}{\makebox(0,0){$\Diamond$}}}
\put(260,299){\raisebox{-.8pt}{\makebox(0,0){$\Diamond$}}}
\put(806,227){\raisebox{-.8pt}{\makebox(0,0){$\Diamond$}}}
\put(1352,299){\raisebox{-.8pt}{\makebox(0,0){$\Diamond$}}}
\put(1394,389){\raisebox{-.8pt}{\makebox(0,0){$\Diamond$}}}
\sbox{\plotpoint}{\rule[-0.400pt]{0.800pt}{0.800pt}}%
\put(1306,722){\makebox(0,0)[r]{F(x)}}
\put(1328.0,722.0){\rule[-0.400pt]{15.899pt}{0.800pt}}
\put(176,509){\usebox{\plotpoint}}
\multiput(177.41,497.95)(0.509,-1.608){19}{\rule{0.123pt}{2.662pt}}
\multiput(174.34,503.48)(13.000,-34.476){2}{\rule{0.800pt}{1.331pt}}
\multiput(190.41,457.93)(0.511,-1.621){17}{\rule{0.123pt}{2.667pt}}
\multiput(187.34,463.47)(12.000,-31.465){2}{\rule{0.800pt}{1.333pt}}
\multiput(202.41,422.48)(0.509,-1.360){19}{\rule{0.123pt}{2.292pt}}
\multiput(199.34,427.24)(13.000,-29.242){2}{\rule{0.800pt}{1.146pt}}
\multiput(215.41,389.25)(0.509,-1.236){19}{\rule{0.123pt}{2.108pt}}
\multiput(212.34,393.63)(13.000,-26.625){2}{\rule{0.800pt}{1.054pt}}
\multiput(228.41,359.02)(0.509,-1.112){19}{\rule{0.123pt}{1.923pt}}
\multiput(225.34,363.01)(13.000,-24.009){2}{\rule{0.800pt}{0.962pt}}
\multiput(241.41,330.97)(0.511,-1.123){17}{\rule{0.123pt}{1.933pt}}
\multiput(238.34,334.99)(12.000,-21.987){2}{\rule{0.800pt}{0.967pt}}
\multiput(253.41,306.29)(0.509,-0.905){19}{\rule{0.123pt}{1.615pt}}
\multiput(250.34,309.65)(13.000,-19.647){2}{\rule{0.800pt}{0.808pt}}
\multiput(266.41,283.81)(0.509,-0.823){19}{\rule{0.123pt}{1.492pt}}
\multiput(263.34,286.90)(13.000,-17.903){2}{\rule{0.800pt}{0.746pt}}
\multiput(279.41,263.32)(0.509,-0.740){19}{\rule{0.123pt}{1.369pt}}
\multiput(276.34,266.16)(13.000,-16.158){2}{\rule{0.800pt}{0.685pt}}
\multiput(292.41,244.74)(0.511,-0.671){17}{\rule{0.123pt}{1.267pt}}
\multiput(289.34,247.37)(12.000,-13.371){2}{\rule{0.800pt}{0.633pt}}
\multiput(304.41,229.34)(0.509,-0.574){19}{\rule{0.123pt}{1.123pt}}
\multiput(301.34,231.67)(13.000,-12.669){2}{\rule{0.800pt}{0.562pt}}
\multiput(316.00,217.08)(0.492,-0.509){19}{\rule{1.000pt}{0.123pt}}
\multiput(316.00,217.34)(10.924,-13.000){2}{\rule{0.500pt}{0.800pt}}
\multiput(329.00,204.08)(0.539,-0.512){15}{\rule{1.073pt}{0.123pt}}
\multiput(329.00,204.34)(9.774,-11.000){2}{\rule{0.536pt}{0.800pt}}
\multiput(341.00,193.08)(0.654,-0.514){13}{\rule{1.240pt}{0.124pt}}
\multiput(341.00,193.34)(10.426,-10.000){2}{\rule{0.620pt}{0.800pt}}
\multiput(354.00,183.08)(0.847,-0.520){9}{\rule{1.500pt}{0.125pt}}
\multiput(354.00,183.34)(9.887,-8.000){2}{\rule{0.750pt}{0.800pt}}
\multiput(367.00,175.07)(1.244,-0.536){5}{\rule{1.933pt}{0.129pt}}
\multiput(367.00,175.34)(8.987,-6.000){2}{\rule{0.967pt}{0.800pt}}
\multiput(380.00,169.07)(1.132,-0.536){5}{\rule{1.800pt}{0.129pt}}
\multiput(380.00,169.34)(8.264,-6.000){2}{\rule{0.900pt}{0.800pt}}
\put(392,161.34){\rule{2.800pt}{0.800pt}}
\multiput(392.00,163.34)(7.188,-4.000){2}{\rule{1.400pt}{0.800pt}}
\put(405,157.84){\rule{3.132pt}{0.800pt}}
\multiput(405.00,159.34)(6.500,-3.000){2}{\rule{1.566pt}{0.800pt}}
\put(418,155.34){\rule{3.132pt}{0.800pt}}
\multiput(418.00,156.34)(6.500,-2.000){2}{\rule{1.566pt}{0.800pt}}
\put(431,153.84){\rule{2.891pt}{0.800pt}}
\multiput(431.00,154.34)(6.000,-1.000){2}{\rule{1.445pt}{0.800pt}}
\put(443,152.84){\rule{3.132pt}{0.800pt}}
\multiput(443.00,153.34)(6.500,-1.000){2}{\rule{1.566pt}{0.800pt}}
\put(456,152.84){\rule{3.132pt}{0.800pt}}
\multiput(456.00,152.34)(6.500,1.000){2}{\rule{1.566pt}{0.800pt}}
\put(469,153.84){\rule{2.891pt}{0.800pt}}
\multiput(469.00,153.34)(6.000,1.000){2}{\rule{1.445pt}{0.800pt}}
\put(481,155.34){\rule{3.132pt}{0.800pt}}
\multiput(481.00,154.34)(6.500,2.000){2}{\rule{1.566pt}{0.800pt}}
\put(494,157.34){\rule{3.132pt}{0.800pt}}
\multiput(494.00,156.34)(6.500,2.000){2}{\rule{1.566pt}{0.800pt}}
\put(507,159.84){\rule{3.132pt}{0.800pt}}
\multiput(507.00,158.34)(6.500,3.000){2}{\rule{1.566pt}{0.800pt}}
\put(520,162.84){\rule{2.891pt}{0.800pt}}
\multiput(520.00,161.34)(6.000,3.000){2}{\rule{1.445pt}{0.800pt}}
\put(532,165.84){\rule{3.132pt}{0.800pt}}
\multiput(532.00,164.34)(6.500,3.000){2}{\rule{1.566pt}{0.800pt}}
\put(545,169.34){\rule{2.800pt}{0.800pt}}
\multiput(545.00,167.34)(7.188,4.000){2}{\rule{1.400pt}{0.800pt}}
\put(558,172.84){\rule{3.132pt}{0.800pt}}
\multiput(558.00,171.34)(6.500,3.000){2}{\rule{1.566pt}{0.800pt}}
\put(571,176.34){\rule{2.600pt}{0.800pt}}
\multiput(571.00,174.34)(6.604,4.000){2}{\rule{1.300pt}{0.800pt}}
\put(583,180.34){\rule{2.800pt}{0.800pt}}
\multiput(583.00,178.34)(7.188,4.000){2}{\rule{1.400pt}{0.800pt}}
\put(596,184.34){\rule{2.800pt}{0.800pt}}
\multiput(596.00,182.34)(7.188,4.000){2}{\rule{1.400pt}{0.800pt}}
\put(609,188.34){\rule{2.600pt}{0.800pt}}
\multiput(609.00,186.34)(6.604,4.000){2}{\rule{1.300pt}{0.800pt}}
\put(621,192.34){\rule{2.800pt}{0.800pt}}
\multiput(621.00,190.34)(7.188,4.000){2}{\rule{1.400pt}{0.800pt}}
\put(634,196.34){\rule{2.800pt}{0.800pt}}
\multiput(634.00,194.34)(7.188,4.000){2}{\rule{1.400pt}{0.800pt}}
\put(647,200.34){\rule{2.800pt}{0.800pt}}
\multiput(647.00,198.34)(7.188,4.000){2}{\rule{1.400pt}{0.800pt}}
\put(660,203.84){\rule{2.891pt}{0.800pt}}
\multiput(660.00,202.34)(6.000,3.000){2}{\rule{1.445pt}{0.800pt}}
\put(672,207.34){\rule{2.800pt}{0.800pt}}
\multiput(672.00,205.34)(7.188,4.000){2}{\rule{1.400pt}{0.800pt}}
\put(685,210.84){\rule{3.132pt}{0.800pt}}
\multiput(685.00,209.34)(6.500,3.000){2}{\rule{1.566pt}{0.800pt}}
\put(698,213.84){\rule{3.132pt}{0.800pt}}
\multiput(698.00,212.34)(6.500,3.000){2}{\rule{1.566pt}{0.800pt}}
\put(711,216.34){\rule{2.891pt}{0.800pt}}
\multiput(711.00,215.34)(6.000,2.000){2}{\rule{1.445pt}{0.800pt}}
\put(723,218.34){\rule{3.132pt}{0.800pt}}
\multiput(723.00,217.34)(6.500,2.000){2}{\rule{1.566pt}{0.800pt}}
\put(736,220.34){\rule{3.132pt}{0.800pt}}
\multiput(736.00,219.34)(6.500,2.000){2}{\rule{1.566pt}{0.800pt}}
\put(749,221.84){\rule{2.891pt}{0.800pt}}
\multiput(749.00,221.34)(6.000,1.000){2}{\rule{1.445pt}{0.800pt}}
\put(761,223.34){\rule{3.132pt}{0.800pt}}
\multiput(761.00,222.34)(6.500,2.000){2}{\rule{1.566pt}{0.800pt}}
\put(787,224.84){\rule{3.132pt}{0.800pt}}
\multiput(787.00,224.34)(6.500,1.000){2}{\rule{1.566pt}{0.800pt}}
\put(774.0,226.0){\rule[-0.400pt]{3.132pt}{0.800pt}}
\put(812,224.84){\rule{3.132pt}{0.800pt}}
\multiput(812.00,225.34)(6.500,-1.000){2}{\rule{1.566pt}{0.800pt}}
\put(800.0,227.0){\rule[-0.400pt]{2.891pt}{0.800pt}}
\put(838,223.34){\rule{3.132pt}{0.800pt}}
\multiput(838.00,224.34)(6.500,-2.000){2}{\rule{1.566pt}{0.800pt}}
\put(851,221.84){\rule{2.891pt}{0.800pt}}
\multiput(851.00,222.34)(6.000,-1.000){2}{\rule{1.445pt}{0.800pt}}
\put(863,220.34){\rule{3.132pt}{0.800pt}}
\multiput(863.00,221.34)(6.500,-2.000){2}{\rule{1.566pt}{0.800pt}}
\put(876,218.34){\rule{3.132pt}{0.800pt}}
\multiput(876.00,219.34)(6.500,-2.000){2}{\rule{1.566pt}{0.800pt}}
\put(889,216.34){\rule{2.891pt}{0.800pt}}
\multiput(889.00,217.34)(6.000,-2.000){2}{\rule{1.445pt}{0.800pt}}
\put(901,213.84){\rule{3.132pt}{0.800pt}}
\multiput(901.00,215.34)(6.500,-3.000){2}{\rule{1.566pt}{0.800pt}}
\put(914,210.84){\rule{3.132pt}{0.800pt}}
\multiput(914.00,212.34)(6.500,-3.000){2}{\rule{1.566pt}{0.800pt}}
\put(927,207.34){\rule{2.800pt}{0.800pt}}
\multiput(927.00,209.34)(7.188,-4.000){2}{\rule{1.400pt}{0.800pt}}
\put(940,203.84){\rule{2.891pt}{0.800pt}}
\multiput(940.00,205.34)(6.000,-3.000){2}{\rule{1.445pt}{0.800pt}}
\put(952,200.34){\rule{2.800pt}{0.800pt}}
\multiput(952.00,202.34)(7.188,-4.000){2}{\rule{1.400pt}{0.800pt}}
\put(965,196.34){\rule{2.800pt}{0.800pt}}
\multiput(965.00,198.34)(7.188,-4.000){2}{\rule{1.400pt}{0.800pt}}
\put(978,192.34){\rule{2.800pt}{0.800pt}}
\multiput(978.00,194.34)(7.188,-4.000){2}{\rule{1.400pt}{0.800pt}}
\put(991,188.34){\rule{2.600pt}{0.800pt}}
\multiput(991.00,190.34)(6.604,-4.000){2}{\rule{1.300pt}{0.800pt}}
\put(1003,184.34){\rule{2.800pt}{0.800pt}}
\multiput(1003.00,186.34)(7.188,-4.000){2}{\rule{1.400pt}{0.800pt}}
\put(1016,180.34){\rule{2.800pt}{0.800pt}}
\multiput(1016.00,182.34)(7.188,-4.000){2}{\rule{1.400pt}{0.800pt}}
\put(1029,176.34){\rule{2.600pt}{0.800pt}}
\multiput(1029.00,178.34)(6.604,-4.000){2}{\rule{1.300pt}{0.800pt}}
\put(1041,172.84){\rule{3.132pt}{0.800pt}}
\multiput(1041.00,174.34)(6.500,-3.000){2}{\rule{1.566pt}{0.800pt}}
\put(1054,169.34){\rule{2.800pt}{0.800pt}}
\multiput(1054.00,171.34)(7.188,-4.000){2}{\rule{1.400pt}{0.800pt}}
\put(1067,165.84){\rule{3.132pt}{0.800pt}}
\multiput(1067.00,167.34)(6.500,-3.000){2}{\rule{1.566pt}{0.800pt}}
\put(1080,162.84){\rule{2.891pt}{0.800pt}}
\multiput(1080.00,164.34)(6.000,-3.000){2}{\rule{1.445pt}{0.800pt}}
\put(1092,159.84){\rule{3.132pt}{0.800pt}}
\multiput(1092.00,161.34)(6.500,-3.000){2}{\rule{1.566pt}{0.800pt}}
\put(1105,157.34){\rule{3.132pt}{0.800pt}}
\multiput(1105.00,158.34)(6.500,-2.000){2}{\rule{1.566pt}{0.800pt}}
\put(1118,155.34){\rule{3.132pt}{0.800pt}}
\multiput(1118.00,156.34)(6.500,-2.000){2}{\rule{1.566pt}{0.800pt}}
\put(1131,153.84){\rule{2.891pt}{0.800pt}}
\multiput(1131.00,154.34)(6.000,-1.000){2}{\rule{1.445pt}{0.800pt}}
\put(1143,152.84){\rule{3.132pt}{0.800pt}}
\multiput(1143.00,153.34)(6.500,-1.000){2}{\rule{1.566pt}{0.800pt}}
\put(1156,152.84){\rule{3.132pt}{0.800pt}}
\multiput(1156.00,152.34)(6.500,1.000){2}{\rule{1.566pt}{0.800pt}}
\put(1169,153.84){\rule{2.891pt}{0.800pt}}
\multiput(1169.00,153.34)(6.000,1.000){2}{\rule{1.445pt}{0.800pt}}
\put(1181,155.34){\rule{3.132pt}{0.800pt}}
\multiput(1181.00,154.34)(6.500,2.000){2}{\rule{1.566pt}{0.800pt}}
\put(1194,157.84){\rule{3.132pt}{0.800pt}}
\multiput(1194.00,156.34)(6.500,3.000){2}{\rule{1.566pt}{0.800pt}}
\put(1207,161.34){\rule{2.800pt}{0.800pt}}
\multiput(1207.00,159.34)(7.188,4.000){2}{\rule{1.400pt}{0.800pt}}
\multiput(1220.00,166.39)(1.132,0.536){5}{\rule{1.800pt}{0.129pt}}
\multiput(1220.00,163.34)(8.264,6.000){2}{\rule{0.900pt}{0.800pt}}
\multiput(1232.00,172.39)(1.244,0.536){5}{\rule{1.933pt}{0.129pt}}
\multiput(1232.00,169.34)(8.987,6.000){2}{\rule{0.967pt}{0.800pt}}
\multiput(1245.00,178.40)(0.847,0.520){9}{\rule{1.500pt}{0.125pt}}
\multiput(1245.00,175.34)(9.887,8.000){2}{\rule{0.750pt}{0.800pt}}
\multiput(1258.00,186.40)(0.654,0.514){13}{\rule{1.240pt}{0.124pt}}
\multiput(1258.00,183.34)(10.426,10.000){2}{\rule{0.620pt}{0.800pt}}
\multiput(1271.00,196.40)(0.539,0.512){15}{\rule{1.073pt}{0.123pt}}
\multiput(1271.00,193.34)(9.774,11.000){2}{\rule{0.536pt}{0.800pt}}
\multiput(1283.00,207.41)(0.492,0.509){19}{\rule{1.000pt}{0.123pt}}
\multiput(1283.00,204.34)(10.924,13.000){2}{\rule{0.500pt}{0.800pt}}
\multiput(1297.41,219.00)(0.509,0.574){19}{\rule{0.123pt}{1.123pt}}
\multiput(1294.34,219.00)(13.000,12.669){2}{\rule{0.800pt}{0.562pt}}
\multiput(1310.41,234.00)(0.511,0.671){17}{\rule{0.123pt}{1.267pt}}
\multiput(1307.34,234.00)(12.000,13.371){2}{\rule{0.800pt}{0.633pt}}
\multiput(1322.41,250.00)(0.509,0.740){19}{\rule{0.123pt}{1.369pt}}
\multiput(1319.34,250.00)(13.000,16.158){2}{\rule{0.800pt}{0.685pt}}
\multiput(1335.41,269.00)(0.509,0.823){19}{\rule{0.123pt}{1.492pt}}
\multiput(1332.34,269.00)(13.000,17.903){2}{\rule{0.800pt}{0.746pt}}
\multiput(1348.41,290.00)(0.509,0.905){19}{\rule{0.123pt}{1.615pt}}
\multiput(1345.34,290.00)(13.000,19.647){2}{\rule{0.800pt}{0.808pt}}
\multiput(1361.41,313.00)(0.511,1.123){17}{\rule{0.123pt}{1.933pt}}
\multiput(1358.34,313.00)(12.000,21.987){2}{\rule{0.800pt}{0.967pt}}
\multiput(1373.41,339.00)(0.509,1.112){19}{\rule{0.123pt}{1.923pt}}
\multiput(1370.34,339.00)(13.000,24.009){2}{\rule{0.800pt}{0.962pt}}
\multiput(1386.41,367.00)(0.509,1.236){19}{\rule{0.123pt}{2.108pt}}
\multiput(1383.34,367.00)(13.000,26.625){2}{\rule{0.800pt}{1.054pt}}
\multiput(1399.41,398.00)(0.509,1.360){19}{\rule{0.123pt}{2.292pt}}
\multiput(1396.34,398.00)(13.000,29.242){2}{\rule{0.800pt}{1.146pt}}
\multiput(1412.41,432.00)(0.511,1.621){17}{\rule{0.123pt}{2.667pt}}
\multiput(1409.34,432.00)(12.000,31.465){2}{\rule{0.800pt}{1.333pt}}
\multiput(1424.41,469.00)(0.509,1.608){19}{\rule{0.123pt}{2.662pt}}
\multiput(1421.34,469.00)(13.000,34.476){2}{\rule{0.800pt}{1.331pt}}
\put(825.0,226.0){\rule[-0.400pt]{3.132pt}{0.800pt}}
\end{picture}

\end{center}
\caption{Plot of $f(x)$, samples, and $F(x)$}\label{fig:exp.ps}
\end{figure}

We use \texttt{interpolate} to compute a polynomial $F(x)$ for the
samples by calling Perl. In Figure~\ref{fig:exp.pl} you see the
contents of \texttt{exp.pl}.
% in Figure~\ref{fig:sample.dat} you see the contents of \texttt{sample.dat}.
%
As can be seen in Figure~\ref{fig:exp.ps}, the derived polynomial
$F(x)$ swings widely when compared with $f(x)$. 
%
This means that you can not expect any set of sample to give good
values between the samples.

\begin{figure}[tb]
\begin{center}
\begin{minipage}{0.8\hsize}
\verbatiminput{exp.pl}
\end{minipage}
\end{center}
\caption{The file \texttt{exp.pl}}\label{fig:exp.pl}
\end{figure}

% \begin{figure}[t]
% \begin{center}
% \begin{minipage}{0.3\hsize}
% \verbatiminput{sample.dat}
% \end{minipage}
% \end{center}
% \caption{The file \texttt{sample.dat}}\label{fig:sample.dat}
% \end{figure}

\end{document}
